\documentclass[a4paper]{article}

\renewcommand{\labelenumii}{\theenumii}
\renewcommand{\theenumii}{\theenumi.\arabic{enumii}.}

\usepackage{fullpage} % Package to use full page
\usepackage{parskip} % Package to tweak paragraph skipping
\usepackage{tikz} % Package for drawing
\usepackage{amsmath}
\usepackage{amssymb,amsthm}
\usepackage{mathtools}
\usepackage{hyperref}
\usepackage{enumitem}
\usepackage{mathtools}
\usepackage{empheq}
\usepackage{amsfonts}
\usepackage{gensymb}
\usepackage{float}
\usepackage{subcaption}
\usepackage{placeins}


\newcommand*\widefbox[1]{\fbox{\hspace{2em}#1\hspace{2em}}}
%======================================== Title===========================================
\newcommand{\horrule}[1]{\rule{\linewidth}{#1}} 	% Horizontal rule
\title{
		\vspace{-0.6in} 	
		\usefont{OT1}{bch}{b}{n}
		\normalfont \normalsize \textsc{University of California Santa Cruz} \\ [10pt]
		\horrule{0.5pt} \\[0.4cm]
		\huge CMPE264 Project 2: Sweeping Plane Stereo \\
		\horrule{2pt} \\[0.5cm]
}
\author{
		\normalfont 								
        Aaron Hunter\\  Carlos Espinosa\\[-3pt]		\normalsize
}
%====================================Begin document=======================================
\begin{document}
\maketitle
%====================================Introduction=======================================
\section{Introduction}
In this report we document the process of generating a depth map of a scene by taking two images from different points of view.  To calculate the depth we triangulate between points that are visible in both images.  To perform the triangulation, however, we must determine both the relative camera orientations as well as the intrinsic camera parameters.  

The camera model (i.e., the intrinsic camera parameters) is developed through a calibration of a known target (in this case a checkerboard image) done in part 1. 

To find the relative camera pose we first need to capture two images of the same scene from different vantage points.  In Part 2 we construct a scene that comprised several different parallel planes with distinctive features. It is important to ensure that both images are translated and rotated enough to generate a large enough disparity between the images but still captures the same scene.  If the distance and rotation is too great, however, the homography applied to the images appears very distorted and doesn't provide a useful depth map.  

The relative orientation of the two cameras, known as the relative camera pose, is composed of the distance between the optical centers of the two camera positions and the relative orientation of the optical axis. This is performed in Part 3. The distance between the optical centers is known as the baseline and the orientation of the optical axes is a rotation matrix which is composed of the rotations around three orthogonal axes. To find the relative camera pose, we use functions in the OpenCV library to identify features common to both images.  We then use a matching function that determines the quality of the features, in other words, the confidence that both points actually correspond to a given surface point in the world.  Once identified we use another OpenCV function to determine the Essential Matrix, which relates the points in one camera reference frame to the other.  To confirm the fit we project the points found in one image onto the points found on the other image. If they coincide closely then we can have confidence in the determination of the relative camera pose.  

Finally, in Part 4 we implement the sweeping-stereo approach to calculating a depth map.  This is done by determining the depth of each feature point using the relative camera pose.  We then calculate the minimum and maximum depth in the scene and divide the three dimensional volume into 20 discrete planes spanning the depth of the image. We then warp one image using the plane parameters to generate the homography that translates a planar image in one reference system to the other. For each plane we determine which pixels coincide using an absolute difference between the images and finding the minimum value--where these pixels overlap with the reference image we can safely assign a depth to those pixels.  This is done for every pixel over all 20 warped images until the minimum is found. Finally, we convert these depths values into a gray scale eight bit integer and display a depth map.  Note the depth is relative to the baseline of the two vantage points.
%====================================Part 1=======================================
\section{Camera calibration}
In this section we perform the camera calibration by taking many (some websites recommend more than 20) images of a known target.  We selected a fixed focal length lens to minimize any possible variations between the calibration images and the final scene images. The camera used is a Fuji X-E1 with a 60mm prime lens set to f/8.  The target is a 6x9 checkerboard target taken from OpenCV.org, taped it to a flat board and captured images of it at multiple orientations.  Rather than move the camera around, we fixed the camera positio
1. The pictures you took of the chessboard pattern
2. The intrinsic matrix K
3. The radial distortion coefficients
4. The reprojection mean square error
%====================================Part 2=======================================
\section{Scene Images}
%====================================Part 3=======================================
\section{Relative Camera Pose}
1.Epipolar lines
2. Write down the matrix 𝑅𝑅, 𝒓𝑅 you found. 𝐿
 3. Show the re-projected feature points on the first image
%====================================Part 4=======================================
\section{Plane Sweeping Stereo}

Deliverables:
min and max depths found
N=20 warped images
Depth image (grayscale)
%====================================Conclusions=======================================
\section{Conclusions}






\end{document}